\documentclass{book}
\usepackage{ctex}
\usepackage[margin=0.8in]{geometry}
\geometry{top=2.5cm}
\usepackage{amsmath}
\usepackage{caption}
\usepackage{indentfirst}
\usepackage{graphicx}
\usepackage{subfigure}
\usepackage{amssymb}
\usepackage{cuted}
\usepackage{color}
\usepackage[dvipsnames]{xcolor}
\usepackage{fancyhdr}
\usepackage{xeCJK}
\usepackage{titlesec}
\definecolor{titleBlue}{RGB}{30,59,150}
\titleformat{\chapter}[block]{\huge\bfseries\color{titleBlue}}{第 \thechapter 章}{1em}{}
\titleformat{\section}[block]{\Large\bfseries\color{titleBlue}}{\thesection}{1em}{}
\titleformat{\subsection}[block]{\large\bfseries\color{titleBlue}}{\thesubsection}{1em}{}
\titlespacing*{\section} {0pt}{10pt}{0pt}
% \renewcommand{\chaptermark}[1]{\markboth{第\,\thechapter\,章\quad #1}{}}

\pagestyle{fancy}
\renewcommand{\chaptermark}[1]{\markboth{\CJKfamily{hei} \color{titleBlue}{第 \thechapter 章\quad #1} }{}}
\renewcommand{\sectionmark}[1]{\markright{\CJKfamily{hei} \color{titleBlue}  \thesection \quad #1}{}}

\fancyhead{} % clear all fields
\fancyhead[LO]{\CJKfamily{hei} \bfseries \color{titleBlue}{\rightmark}}
\fancyhead[RO]{\CJKfamily{hei} \bfseries \color{titleBlue}  \thepage}
\fancyhead[LE]{\CJKfamily{hei} \bfseries \color{titleBlue}  \thepage}
\fancyhead[RE]{\CJKfamily{hei} \bfseries \color{titleBlue}{\leftmark}}

\fancyfoot{}

\renewcommand{\headrulewidth}{0pt}
\renewcommand{\footrulewidth}{0pt}


\setlength{\parindent}{2em}
\renewcommand {\thetable} {\thechapter{}.\arabic{table}}
\renewcommand {\thefigure} {\thechapter{}.\arabic{figure}}
\numberwithin{equation}{chapter}

\newcommand {\bx} {\boldsymbol{\mathrm{x}}}
\newcommand {\bw} {\boldsymbol{\mathrm{w}}}
\newcommand {\sfx} {\boldsymbol{\mathsf{x}}}
\newcommand {\sft} {\boldsymbol{\mathsf{t}}}
\newcommand {\sfy} {\boldsymbol{\mathsf{y}}}
\newcommand {\rmT} {\mathrm{T}}
\newcommand {\rmd} {\mathrm{d}}
\newcommand {\bfMu} {\boldsymbol{\mu}}
\newcommand {\bfAl} {\boldsymbol{\alpha}}
\newcommand {\bfSigma} {\boldsymbol{\Sigma}}
\newcommand {\bfLambda} {\boldsymbol{\Lambda}}
\newcommand {\bfPhi} {\boldsymbol{\Phi}}
\newcommand {\bfphi} {\boldsymbol{\phi}}
\newcommand {\bfeta} {\boldsymbol{\eta}}
\newcommand {\calD} {\mathcal{D}}
\newcommand {\calN} {\mathcal{N}}
\newcommand {\calR} {\mathcal{R}}
\newcommand {\insertline} {\noindent{\color{red} \rule[5pt]{\textwidth}{0.1em}}}

\author{张括嘉 \\ 东北大学机器人科学与工程学院}

\begin{document}
\title{Pattern Recognition and Machine Learning 中文版}
\date{}
% \maketitle
	\chapter{绪 \quad 论}
	\noindent\rule[0.25\baselineskip]{\textwidth}{1pt}
	\renewcommand {\thetable} {\thechapter{}.\arabic{table}}
	\renewcommand {\thefigure} {\thechapter{}.\arabic{figure}}
	
	\chapter{概率分布}
	\noindent\rule[0.25\baselineskip]{\textwidth}{1pt}
	\renewcommand {\thetable} {\thechapter{}.\arabic{table}}
	\renewcommand {\thefigure} {\thechapter{}.\arabic{figure}}
	
	\chapter{线性回归模型}
	\noindent\rule[0.25\baselineskip]{\textwidth}{1pt}
	\renewcommand {\thetable} {\thechapter{}.\arabic{table}}
	\renewcommand {\thefigure} {\thechapter{}.\arabic{figure}}
	
	\chapter{线性分类模型}
	\noindent\rule[0.25\baselineskip]{\textwidth}{1pt}
	\renewcommand {\thetable} {\thechapter{}.\arabic{table}}
	\renewcommand {\thefigure} {\thechapter{}.\arabic{figure}}

	\chapter{神经网络}
	\noindent\rule[0.25\baselineskip]{\textwidth}{1pt}
	\renewcommand {\thetable} {\thechapter{}.\arabic{table}}
	\renewcommand {\thefigure} {\thechapter{}.\arabic{figure}}
	\textnormal{
	
	}
\end{document}